\documentclass[12pt]{article}
\usepackage[utf8]{inputenc}
\usepackage{indentfirst}
%\usepackage{amsfonts}
\usepackage[portuges,brazil]{babel}
\usepackage[top=3cm, bottom=3cm, left=2cm, right=2cm]{geometry}
\usepackage{amsmath}
\usepackage{mathptmx}
\usepackage{setspace}
\onehalfspacing

\newtheorem{method}{Método}[section]

\begin{document}
\title{Implementação paralela da heurística de reordenamento
de Cuthill-McKee reversa}
\maketitle

{\bf
\begin{center}
Projeto de Pesquisa -- Iniciação Científica\\
Instituto de Matemática, Estatística e Computação Científica\\
Universidade Estadual de Campinas -- UNICAMP
\end{center}
}

{\bf Bolsista:} Raniere Gaia Costa da Silva

{\bf Orientador:} Aurelio Ribeiro Leite de Oliveira

\begin{abstract}
Cada iteração de um método de pontos interiores envolve a solução
de um ou mais sistemas lineares. Este é o passo mais
caro destes métodos do ponto de vista computacional.
Em aplicações reais esses sistemas quase
sempre possuem dimensões elevadas e alto grau de esparsidade.
Usualmente métodos diretos são utilizados para resolver esses sistemas .
A abordagem mais utilizada nas implementações existentes é a fatoração
de Cholesky no sistema de equações normais.
No entanto, em problemas de grande porte, é necessária o reordenamento
eficiente da matriz do sistema linear para evitar o enchimento excessivo
na fatoração e é conveniente tirar proveito do uso de computação paralela.
A grande maioria das implementações existentes não utiliza computação paralela e
faz uso do reordenamento clássico pela heurística de mínimo grau.
Recentemente, surgiram diversas novas heurísticas e outras foram revitalizadas.
Dentre essas, a de Cuthill-McKee reversa se destaca como uma das mais
promissoras por oferecer um bom equilíbrio entre o paralelismo e a eficiência do
reordenamento.
Neste projeto serão comparadas as heurísticas de mínimo grau e Cuthill-McKee
reversa paralelizada tanto com relação ao tempo total de reordenamento quanto
na quantidade de enchimento obtida por cada uma. Os testes serão realizados
no contexto dos métodos de pontos interiores para programação linear
onde uma sequência de sistemas lineares com o mesmo reordenamento
deve ser resolvida.
\end{abstract}

\section{Introdução}
Desde o surgimento dos métodos de pontos interiores para programação linear,
códigos computacionais sofisticados baseados nessas ideias vêm se firmando como
alternativas eficientes para solução de problemas de grande porte com
estruturas genéricas \cite{ARVK89,BCO06,CMWW96,Go96,LMS92,OS03,OL91}.

Nos últimos anos, apareceram novos códigos computacionais sofisticados também
baseados nessas ideias mas que aproveitam a estrutura de uma classe de problemas
e/ou fazem uso da computação paralela possibilitada pelos computadores atuais
que possuem vários núcleos \cite{karypis1994parallel,gondzio2006direct,gondzio2003parallel}.

Cada iteração de um método de pontos interiores envolve a solução
de um ou mais sistemas lineares \cite{Go96,LMS92,Me92b}. Este é o passo mais
caro destes métodos do ponto de vista computacional.
Em aplicações reais esses sistemas quase
sempre possuem dimensões elevadas e alto grau de esparsidade.

Usualmente métodos diretos são utilizados para resolver esses sistemas.
A abordagem mais utilizada nas implementações existentes é a fatoração
de Cholesky no sistema de equações normais \cite{ARVK89,CMWW96,Go96,LMS92}.

Para calcular a fatoração de Cholesky, é necessário inicialmente reordenar
as linhas da matriz de restrições, o que equivale a reordenar de forma
simétrica linhas e colunas da matriz dos sistemas lineares a serem resolvido.
Este reordenamento é realizado uma única vez e utilizado para todos os
sistemas lineares.

A heurística de mínimo grau \cite{GL89} é a mais utilizada na grande maioria de
implementações existentes. Diferentes heurísticas começaram a ser utilizadas
mais recentemente como Cuthill-McKee reverso \cite{CM69}, Mínimo grau
múltiplo \cite{Li85}, Mínimo grau aproximado \cite{ADD96}, {\it Nested
Dissection}) \cite{RH98}.

Em um de seus trabalhos recentes, Frederic Gibou e Chohong Min
\cite{gibou2012performance} obtiveram bons resultados ao resolver sistemas
lineares utilizando uma versão paralela da heurística Cuthill-McKee uma vez que
esta oferece um bom equilíbrio entre o paralelismo e a eficiência do
reordenamento.

Este projeto consiste na implementação paralela da heurística de reordenamento
Cuthill-McKee reversa e comparar sua eficiência para problemas de grande
porte com a heurística clássica de mínimo grau. A comparação será feita
com respeito ao tempo computacional para realizar o reordenamento e também
com relação ao número de elementos obtido na fatoração da matriz reordenada.
Os detalhes desta proposta são apresentados nas seções a seguir.

\section{Programação Linear}
\subsection{O Problema na Forma Padrão}
Na descrição deste projeto consideraremos o problema de programação linear na
forma padrão:
\begin{displaymath}
\begin{array}{lll}
\text{minimizar} & c^tx & \\
\text{sujeito a} & Ax = b, & x \ge 0,
\end{array}
\end{displaymath}
onde $A$ $m \times n$  é uma matriz de posto completo $m$
e $c$, $b$ e $x$ são vetores colunas de dimensão apropriada.
Associado a este problema temos o problema dual:
\begin{displaymath}
\begin{array}{lll}
\text{maximizar} & b^ty &\\
\text{sujeito a} & A^ty + z = c, & z \ge 0,
\end{array}
\end{displaymath}
onde $y$ é um vetor coluna de dimensão $m$ de variáveis livres e $z$ é o vetor
coluna de dimensão $n$ de variáveis de folga duais. O {\it gap} dual é
dado por $\gamma= c^tx - b^ty$ que se reduz a $\gamma= x^tz$ para
pontos primais e duais factíveis.

A direção afim nos métodos de pontos interiores primais-duais é dada por
\cite{MAR90,Wr96}:
\begin{eqnarray} \label{BS}
\left(\begin{array}{llc}
0 & I & A^{t}\\
Z & X & 0\\
A & 0 & 0\\
\end{array} \right)
\left(\begin{array}{r}
\Delta \tilde{x}\\
\Delta \tilde{z}\\
\Delta \tilde{y}
\end{array} \right) &=
\left(\begin{array}{c}
r_d\\
r_a\\
r_p
\end{array} \right)
\end{eqnarray}
onde $X = diag(x)$, $Z = diag(z)$ e os resíduos são dados por $r_p = b - Ax$,
$r_d = c - A^ty - z$, $r_a = -XZe$ e $e$ representa o vetor de uns.

Eliminando as variáveis $\Delta \tilde{z}$ de (\ref{BS}) obtemos o sistema
aumentado:
\begin{eqnarray} \label{GAS}
\left(\begin{array}{rc}
-D & A^t\\
 A & O
\end{array} \right)
\left(\begin{array}{r}
\Delta \tilde{x}\\
\Delta \tilde{y}
\end{array} \right) &=
\left(\begin{array}{c}
r_1\\
r_p
\end{array} \right)
\end{eqnarray}
onde $D = X^{-1}Z$.\\

A forma mais utilizada para resolver (\ref{GAS}) consiste em reduzir o
sistema através de eliminação das variáveis $\Delta \tilde{x}$ a um sistema
simétrico definido positivo com a matriz de equações normais
$AD^{-1}A^t$ e aplicar então a fatoração de Cholesky
\cite{ARVK89,CMWW96,Go96}.

A extensão das ideias apresentadas neste projeto para problemas com
variáveis canalizadas é imediata e não será tratada explicitamente na
discussão a seguir embora venham a ser consideradas nas pesquisas.

\subsection{Reordenamento da Matriz do Sistema de Equações Normais}
Nos métodos diretos o sistema de equações normais
$AD^{-1}A^t$ é reordenado através de uma matriz de
permutações $Q$ obtendo $QAD^{-1}A^tQ^t$ onde espera-se
que a matriz reordenada produza menos enchimento que a
matriz original durante a fatoração.

Existem diversas heurísticas de reordenamento na literatura.
Algumas delas são utilizadas com mais frequência nos métodos
de pontos interiores. Dentre elas destacamos a heurística
Cuthill-McKee reverso \cite{CM69}.

Esta heurística de reordenamento tem como objetivo principal reduzir a largura
de banda de uma matriz simétrica. A largura de banda é dada pela maior distância
de um elemento não nulo à diagonal principal. Desta forma, os elementos são
movidos para posições mais próximas da diagonal.  Uma vez que esta propriedade é
mantida na fatoração de Cholesky correspondente, o reordenamento tende a reduzir
o enchimento.  Isto torna o algoritmo muito usado na prática. Não há garantia de
se encontrar a menor largura de banda possível, mas isso frequentemente ocorre.

Na proposta de Cuthill-McKee é realizada uma minimização local da
largura de banda correspondente a cada linha para reduzir a largura de
banda da matriz como um todo.

Posteriormente, George \cite{Geo71} propôs que o reordenamento obtido pelo
reverso de Cuthill-McKee fosse utilizado pois este último
frequentemente, supera o original em termos de redução do {\it profile},
embora mantenha a mesma largura de banda. Esse reordenamento é conhecido
como Cuthill-McKee reverso.

Considerando um grafo conectado, o problema de reduzir a largura de banda
é convertido em encontrar uma renumeração dos vértices do grafo tal que
a diferença entre os índices seja a mínima.

\section{Objetivos}
O objetivo deste projeto consiste no estudo de computação paralela, da
implementação paralela da heurística de Cuthill-McKee reversa, já estudada pelo
aluno, e sua integração
com a implementação já existente dos métodos de pontos interiores PCx
\cite{CMWW96} que utiliza a heurística de reordenamento de mínimo grau. Testes
computacionais exaustivos serão realizados com problemas de programação linear
de grande porte.

A comparação dos resultados obtidos será realizada quanto ao tempo de
reordenamento, quantidade de elementos não nulos na fatoração de Cholesky
e na influência destes no tempo de solução do problema de programação linear
pelo método de pontos interiores.

\section{Etapas a serem Cumpridas e Cronograma}
\subsection{Etapas}
Os objetivos propostos neste projeto podem ser alcançados pela realização das
seguintes etapas:
\begin{enumerate}
\item Revisão bibliográfica. O aluno deverá ler artigos e capítulos de livros
relacionados com o tema do projeto aprofundando e ampliando seus
conhecimentos em relação à computação paralela.
\item Implementações e testes. O aluno deverá implementar, objetivando
a maior eficiência possível, e realizar extensivos testes computacionais com
problemas reais de grande porte comparando as heurísticas com relação ao tempo
de processamento e memória quantidade de memória necessária na fatoração de
Cholesky.
\end{enumerate}

\subsection{Cronograma}
A tabela a seguir resume as atividades a serem realizadas durante
o programa de iniciação científica.
\begin{table}[!h]
\centering
\caption{Cronograma}
\begin{tabular}{|r|c|c|c|} \hline
Bimestre & 1 & 2 \\
\hline
$1^{\underline{o}}$ & X &  \\
\hline
$2^{\underline{o}}$ &   & X \\
\hline
$3^{\underline{o}}$ &   & X \\
\hline
\end{tabular}
\end{table}

Pretende-se divulgar os resultados deste trabalho em um congresso nacional
da área.

\bibliographystyle{siam}
\bibliography{references}

\end{document}
