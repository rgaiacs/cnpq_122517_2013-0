\documentclass[12pt,a4paper]{article}
\input{pacotes.tex}
\usepackage{biblatex}
\addbibresource{referencias.bib}
\begin{document}
% Identificação
%     Projeto
%     Bolsista / RA
%     Orientador
%     Local de execução
%     Vigência
\title{Implementação paralela da heurística de reordenamento
de Cuthill-McKee reversa}
\author{Raniere Silva\footnote{\url{ra092767@ime.unicamp.br}} \\ Orientando
\and Aurelio Oliveira\footnote{\url{aurelio@ime.unicamp.br}} \\ Orientador}
\maketitle

\tableofcontents

% Introdução
%     Introdução ao assunto: deve ser bastante geral
%     Informações da literatura: tornam a introdução mais específica ao assunto
%     Colocação da questão estudada: especificar os objetivos do trabalho
%     Atividades desenvolvidas: dar uma ideia geral de como foi desenvolvido o
%     trabalho
\input{introducao@relatorio_final.tex}

% Materiais e Métodos
%     Materiais: citar os equipamentos, reagentes e outros tons utilizados,
%     informando fabricante ou fornecedor
%     Métodos: descrever os procedimentos detalhados, que possam ser
%     reproduzidos com os materiais e equipamentos descritos
\section{Implementação do Método e Testes Computacionais}
O Método Cuthill-McKee Reverso Paralelo começou a ser implementado, na
linguagem C e utilizando a biblioteca Open MPI \cite{open_mpi}, pelo autor
deste trabalho como um patch para o PCx
(\url{http://pages.cs.wisc.edu/~swright/PCx/}), e a implementação desenvolvida
encontra-se disponível em
\url{https://github.com/r-gaia-cs/PCx/tree/dev-parallel}.


% Resultados
%     Descrição dos resultados: deve ser clara e objetiva, resumindo os achados
%     principais que serão detalhados em tabelas e figuras.
%     Ilustrações dos resultados: tabelas e figuras são muito importantes; seu
%     número deve ser o menor possível, e elas devem ser construídas com cuidado
%     para incluir todas as informações necessárias com clareza.
%     Tabelas: devem ser numeradas sequencialmente (Tabela 1, Tabela 2, etc).
%     Seu título deve ser informativo, colocado acima e justificado à esquerda.
%     Notas de rodapé (a, b, c...) podem ser colocados diretamente abaixo da
%     mesma.
%     Figuras (fotos, esquemas, gráficos): devem ser numeradas sequencialmente
%     (Figura 1, Figura 2, etc). Seu título deve ser informativo, colocado
%     abaixo e justificado à esquerda, descrevendo o que é mostrado.
%\section{Resultados Obtidos e Conclusão}
Os dados referentes aos testes computacionais encontram-se na
Tabela~\ref{tab:resul} sendo que as colunas ``Lar. Banda'', ``Envelope''
referem-se a matriz $A D^{-1} A^T$ e ``Lar. Banda R.'' e ``Envelope R.''
referem-se a matriz $P A D^{-1} A^T P^T$ em que a matriz de permutação $P$ é
obtida pelo Método Cuthill-McKee Reverso.


% Discussão / Conclusões
%     Descrição dos dados à luz da literatura
%     Descrição de possíveis fontes de erro e seu efeito sobre os dados
%     Se seus experimentos falharam, quais as sugestões para corrigir o
%     problema?
\section{Conclusões}
A implementação em paralela consumiu mais tempo que o esperado e não foi
concluída. Consequentemente, testes computacionais não puderam ser feitos.


% Matéria encaminhada para publicação
%     Quando houver, referir resumos ou artigos científicos publicados ou
%     encaminhados para publicação
\input{publicacoes@relatorio_final.tex}

% Perspectivas de continuidade ou desdobramento do trabalho
%     O projeto foi concluído ou será continuado?
\input{desdobramentos@relatorio_final.tex}

% Outras atividades de interesse universitário
%     Descrever participações em congressos, cursos extra-curriculares, etc
\section{Apresentações em Congressos}
O aluno apresentou o trabalho anterior de iniciação, financiado pelo Conselho
Nacional de Desenvolvimento Científico e Tenológico pelo Processo
128674/2012-3, no XLV Simpósio Brasileiro de Pesquisa Operacional em setembro
de 2013 e no XXI Congresso Interno de Iniciação Científica da UNICAMP em
outubro de 2014.

Um poster será apresentado no X Brazilian Workshop on Continuous Optimization
em março de 2014.


% TODO Para a versao final, comentar as linhas a seguir.
\appendix
% Apoio
%     Citar as agências que financiaram o projeto
\section{Informa\c{c}\~{o}es adicionais}
Este trabalho foi financiado pelo Conselho Nacional de Desenvolvimento
Cient\'{i}fico e Tecnol\'{o}gico pelo Processo 122517/2013-0.

É possível adquirir uma cópia deste trabalho em \url{https://github.com/r-gaia-cs/cnpq_122517_2013-0}.

Este trabalho \'{e} licenciado sob a Licen\c{c}a Creative Commons
Atribui\c{c}\~{a}o 3.0 N\~{a}o Adaptada License. Para ver uma c\'{o}pia desta
licen\c{c}a, visite \url{http://creativecommons.org/licenses/by/3.0/}.
\begin{center}
    \includegraphics{figuras/cc-by.png}
\end{center}

% \input{faq@relatorio_final.tex}
% \section{Implementações}
A seguir enumeramos algumas das implementações disponíveis.

\subsection{Fortran90}
\subsubsection{Independente, John Burkardt}
John Burkardt, do Departamento de Computação Científica da The Florida State
University, implementou o RCM.

Disponível em \url{http://people.sc.fsu.edu/~jburkardt/f_src/rcm/rcm.html} e
sob licença GNU LGPL.

\subsection{C}
\subsubsection{Independente, David Fritzsche}
David Fritzsche implementou o RCM para sua tese de mestrado (``Graph
Theoretical Methods for Preconditioners'', defendida em 2004 no
Departamento de Matemática da Bergische Universitat Wuppertal. 

Disponível em \url{http://math.temple.edu/~daffi/software/rcm/} e sob licença ao
estilo BSD.

\subsubsection{Independente, John Burkardt}
John Burkardt, do Departamento de Computação Científica da The Florida State
University, implementou o RCM.

Disponível em \url{http://people.sc.fsu.edu/~jburkardt/cpp_src/rcm/rcm.html} e
sob licença GNU LGPL.

\subsection{C++}
\subsubsection{The Boost Graph Library (BGL)}
É uma biblioteca para trabalhar com grafos.

Encontra-se disponível em
\url{http://www.boost.org/doc/libs/1_51_0/libs/graph/doc/index.html} e sob
\textit{Boost license}.

\subsection{Python}
\subsubsection{PyOrder}
É uma biblioteca para reordenamento de matrizes esparsas.

Encontra-se disponível em \url{https://github.com/dpo/pyorder} sob licença GNU
LGPL.

\subsubsection{NetworkX}
É uma biblioteca para trabalhar com grafos.

Encontra-se disponível em \url{http://networkx.lanl.gov/index.html} e a
implementação do RCM em
\url{http://networkx.lanl.gov/examples/algorithms/rcm.html?highlight=cuthill}.
Encontra-se sob a licença BSD.

\subsection{GNU Octave/Matlab}
\subsubsection{Independente, Michael Weitzel}
Michael Weitzel implementou o RCM como uma extensão da Meschach
C-library e baseou-se em uma fila de prioridades eficiente e uso de
heurística para determinar o nó pseudo-periférico inicial como
descrito no livro ``Computer Solution of Large Sparse Positive Definite
Systems'' de Alan George
e Joseph W. H. Liu.

Sua implementação faz parte do GNU Octave.

\subsubsection{Independente, John Burkardt}
John Burkardt, do Departamento de Computação Científica da The Florida State
University, implementou o RCM.

Disponível em \url{http://people.sc.fsu.edu/~jburkardt/cpp_src/rcm/rcm.html} e
sob licença GNU LGPL.


% Bibliografia
%     Diversos formatos: definir qual o mais apropriado
%     IMPORTANTE
%     - Não liste se não citar.
%     - Não cite se não listar.
\printbibliography[title=Referências]
\end{document}
